\documentclass[10,a4paperpaper,]{article}

  \title{Replication Report}
  \author{Anna Lohmann\textsuperscript{1}, \and Arjan Huizing
\textsuperscript{2}}
  \date{%
		\textsuperscript{1} Leiden University Medical Center\\%
		\textsuperscript{2} University of some other place~\\[2ex]
		\today
   }
  


\newcommand{\iblue}{008080}
\newcommand{\igray}{d4dbde}

\newlength{\cslhangindent}
\setlength{\cslhangindent}{1.5em}
\newenvironment{CSLReferences}%
  {}%
  {\par}

\include{defs}


\begin{document}


\renewcommand{\contentsname}{Table of Contents}

\renewcommand{\pagename}{Page}


\urlstyle{same}

\maketitle

\subsection*{Abstract}

\texttt{\textless{}a\ summary\ of\ the\ replication\ effort\textgreater{}}
\vskip 2em

\noindent\makebox[\textwidth]{\large Correspondence concerning this replication report should be addressed to:}

\par

\noindent\makebox[\textwidth]{\large primary\_a.l.lohmann@lumc.nl}

\par

\clearpage

\section{Introduction}

This replication report documents the replication attempt of the
simulation study Rhemtulla, M., Brosseau-Liard, P. É., \& Savalei, V.
(2012). When can categorical variables be treated as continuous? A
comparison of robust continuous and categorical SEM estimation methods
under suboptimal conditions. Psychological Methods, 17(3), 354--373.
\url{https://doi.org/10.1037/a0029315}

Following the definition of Rougier et al. (2017) we understand the
replication of a published study as writing and running new code based
on the description provided in the original publication with the aim of
obtaining the same results.

\section{Method}

\subsection{Information basis}

The replication attempt was based on the information provided in the
original manuscript as well as the supplemental material accompanying
the publication. The main text provided a link to the supplements
(\url{http://dx.doi.org/10.1037/a0029315.supp}) which referred to the
website of the publisher where an additional pdf document with extensive
result tables was freely available.
\texttt{\textless{}What\ sources\ were\ used\ to\ obtain\ information?\ The\ original\ article,\ some\ appendix,\ online\ supplements,\ other\ articles\ from\ the\ same\ authors,\ code\ available\ from\ the\ authors\ personal\ website?\textgreater{}}

\subsection{Data Generating Mechanism}

Information provided in the above mentioned sources indicated that the
following simulation factors were systematically varied in a full
factorial design for generating the artificial data.

\begin{longtable}[]{@{}
  >{\raggedright\arraybackslash}p{(\columnwidth - 4\tabcolsep) * \real{0.40}}
  >{\raggedright\arraybackslash}p{(\columnwidth - 4\tabcolsep) * \real{0.13}}
  >{\raggedright\arraybackslash}p{(\columnwidth - 4\tabcolsep) * \real{0.47}}@{}}
\toprule
\begin{minipage}[b]{\linewidth}\raggedright
Simulation factor
\end{minipage} & \begin{minipage}[b]{\linewidth}\raggedright
No.~levels
\end{minipage} & \begin{minipage}[b]{\linewidth}\raggedright
Levels
\end{minipage} \\
\midrule
\endhead
\emph{Fixed} & & \\
CFA model size & 2 & 10 indicators, 20 indicators \\
Underlying distribution & 2 & normal, non-normal \\
Number of categories & 6 & 2,3,4,5,6,7 \\
Threshold symmetry & 5 & symmetry, moderate asymmetry, moderate
asymmetry alternative, extreme asymmetry \\
Sample Size & 4 & 100, 150, 350, 600 \\
\bottomrule
\end{longtable}

Data was generated according to these 480 simulation scenarios. This was
repeated for 1000 repetitions.

The data generating mechanism consisted of two steps. (1) Data was
generated based on the underlying distribution, CFA model and sample
size. (2) The generated data was categorized based on the given category
thresholds corresponding to a given number of categories and threshold
symmetry.

\subsubsection{Underlying distribution, CFA model size and Sample Size}

The original study indicated that ``{[}c{]}ontinous data (normal and
nonnormal) were generated in EQS (Version 6.1; Bentler,2008) using
methods developed by Fleishman (1978) and Vale Maurelli (1983).'' We
emulated this approach using the \texttt{generate()} function from the
\texttt{simsem} package with the parameter \texttt{inDist} set to
\texttt{NULL} in the normal case and to
\texttt{simsem::bindDist(skewness\ =\ 2,\ kurtosis\ =\ 7)} in the
non-normal case. This function also to the CFA model (as matrix \ldots)
as well as the sample size as input constituing the first step of the
data generation.

\subsubsection{Number of categories and Threshold symmetry}

\texttt{\textless{}More\ detail\ of\ how\ factor\ 3\ was\ varied\ and\ implemented\textgreater{}}
After data was generated based on the given CFA model and the underlying
distribution the resulting data was categorized into the number of
categories for the scenario at hand. For each number of categories and
each threshold symmetry Z-scores for category thresholds could be
obtained from the first table of the supplemental material.

\texttt{\textless{}You\ can\ add\ pseudocode\ or\ a\ flowchart\ to\ illustrate\ the\ data\ generation\ or\ the\ entire\ simulation\ design\textgreater{}}

\begin{minipage}{\linewidth}
Data generation can be summarized with the following pseudo code:

\texttt{For 1000 repetitions of each of 480 unique scenarios:}
\begin{itemize}[leftmargin=*] 
    \item[--] \texttt{Sample data according to the given CFA model, sample size as well as underlying distribution of the scenario at hand.}
    \item[--] \texttt{Categorize data into the number of categories for the scenario at hand. Category thresholds depended on the threshold symmetry of a given scenario.}
    \item[--] \texttt{If any sample covariance matrix was not positive definitive, repeat sampling until it is.}
        \item[--] \texttt{Analyse data using a robust ML approach.}
        \item[--] \texttt{Analyse data using a robust ULS approach.}
        \item[--] \texttt{Remove results that ....}
        \item[--] \texttt{Obtain performance measures.}
    \begin{itemize}
      \item[$\ast$] \texttt{Parameter estimates}
      \item[$\ast$] \texttt{Bias.}
      \item[$\ast$] \texttt{Compute ... based on these random elements.}
      \item[$\ast$] \texttt{Determine ... based on mechanism of current scenario.}
    \end{itemize}
    \item[--] \texttt{If some condition is > x:}
    \begin{itemize}
      \item[$\ast$] \texttt{Determine ... \& resample from corresponding ... model.}
    \end{itemize}
    \item[--] \texttt{Apply ...}
\end{itemize}
\end{minipage}
\newpage

\FloatBarrier 

\subsection{Investigated Methods}

The study compares the performance of robust normal theory maximum
likelihood (ML) and robust categorical least squares (ULS) methodology
for estimating confirmatory factor analysis (CFA) with ordinary
variables. The underlying CFA model was fit using each of the two
methods under investigation.
\texttt{\textless{}Describe\ the\ methods\ that\ are\ investigated\ and\ how\ they\ are\ implemented\textgreater{}}

\subsubsection{robust normal theory maximum likelihood (ML)}

\texttt{\textless{}Describe\ how\ the\ first\ method\ is\ defined\ and\ implemented.\ You\ can\ include\ equations\ and\ or\ R\ code.\ If\ applicable,\ mention\ specialized\ R\ packages,\ their\ version\ as\ well\ as,\ parameters\ of\ specific\ functions.\textgreater{}}

\subsubsection{robust categorical least squares (ULS)}

\texttt{\textless{}Describe\ how\ the\ second\ method\ is\ defined\ and\ implemented.\ You\ can\ include\ equations\ and\ or\ R\ code.\ If\ applicable,\ mention\ specialized\ R\ packages,\ their\ version\ as\ well\ as,\ parameters\ of\ specific\ functions.\textgreater{}}

\subsection{Performance measures}

The two methods described above were compared on various performance
measures.

\subsubsection{Convergence Failures}
\subsubsection{Improper solutions}
\subsubsection{Outliers}

The orinal study defined outliers as ``any cases that produced a
standard error greater than 1.'' We implemented this as the robust
standard errors listed in the lavaan fit object.

\subsubsection{Parameter Estimates}
\subsubsection{Parameter Bias}
\subsubsection{Efficiency}
\subsubsection{Relative bias for robust standard errors}
\subsubsection{Coverage}
\subsubsection{Type I error rate}
\subsubsection{Outliers}

\texttt{\textless{}Describe\ which\ performance\ measures\ are\ compared,\ if\ applicable\ mention\ specialized\ R\ packages,\ their\ versions,\ as\ well\ as\ parameter\ settings\ of\ functions.\textgreater{}}

The following flowchart describes the simulation design

\begin{figure}
\includegraphics[width=450pt]{flowchart} \caption{Flow chart of data generating mechanism}\label{fig:unnamed-chunk-1}
\end{figure}

\subsection{Technical implementation}

The original simulation study was carried out in EQS (Version 6.1) as
well as Mplus (Version 6.11). The authors of the original study report
that data generation was carried out in EQS and data analysis was
conducted using both EQS as well as MPlus. However, only results from
the Mplus analysis are reported. Our replication was implemented using
the R programming environment (details regarding software versions can
be obtained from the section Reproducibility Information). The
corresponding R code can be obtained from
\url{https://github.com/replisims/rhemtulla-2012}.

The following table provides an overview of replicator degrees of
freedom, i.e.~decisions that had to be made by the replicators because
of insufficient or contradicting information. Issues were resolved by
discussion among the replicators. Decisions were based on what the
replicators perceived to be the most likely implementation with
likeliness estimated by common practice and/or guideline
recommendations. Wherever feasible multiple interpretations where
implemented.

\begin{longtable}[]{@{}
  >{\raggedright\arraybackslash}p{(\columnwidth - 4\tabcolsep) * \real{0.36}}
  >{\raggedright\arraybackslash}p{(\columnwidth - 4\tabcolsep) * \real{0.36}}
  >{\raggedright\arraybackslash}p{(\columnwidth - 4\tabcolsep) * \real{0.27}}@{}}
\toprule
\begin{minipage}[b]{\linewidth}\raggedright
Issue
\end{minipage} & \begin{minipage}[b]{\linewidth}\raggedright
Replicator decision
\end{minipage} & \begin{minipage}[b]{\linewidth}\raggedright
Justification
\end{minipage} \\
\midrule
\endhead
Data dependence & each scenario is implemented in independently
generated data & Best practice (Burton et al. 2006) \\
\bottomrule
\end{longtable}

\subsection{Data basis for Figures 1 and 2}

\texttt{\textless{}More\ details\ on\ how\ the\ information\ provided\ was\ insufficient,\ unclear\ or\ vague\textgreater{}}
\emph{``Some weird quote from the original article that you could not
make any sense of''} (p.XY) We could not infer whether ``for each
condition'' also included for each model size. We interpreted the text
such that this was the case and hence collapsed the distributions across
both model sizes.

\subsection{Another issue}

\texttt{\textless{}More\ details\ on\ how\ the\ information\ provided\ was\ insufficient,\ unclear\ or\ vague\textgreater{}}
\emph{``Some weird quote from the original article that you could not
make any sense of''} (p.XY)

\section{Results}

\subsection{Replication of result figures}
\subsection{Simulation descriptives}

\texttt{\textless{}Describe\ the\ sampling\ distribution\ if\ any\ of\ the\ simulation\ parameters\ were\ sampled\textgreater{}}
The original study provides descriptives for the simulated data in two
figures. Figure 1 and Figure 2 of the original manuscript

\subsection{Replication of result tables}

\texttt{\textless{}Compare\ any\ tabulated\ data\ to\ the\ original\textgreater{}}
Table 1 presents the ``Skew and Kurtosis of Observed Categorical
Variables by Threshold Distribution, Underlying Distribution, and Number
of Categories'' (p.363). The ``{[}v{]}alues in this table were obtained
by generating samples of size N = 1,000,000 for each condition and
recording the skew and kurtosis of the observed distributions.'' (p.363)
As discussed above we understood ``each condition'' to also refer to the
model size. Our results are hence pooled across model size.

\subsection{Replication of results presented in text form }

While the vast majority of results is presented in the form of figures,
a few outcomes regarding outliers, relative bias of parameter estimates
as well as relative bias of robust standard errors are only communicated
in text form.
\texttt{\textless{}If\ the\ text\ describes\ any\ results\ using\ words\ describe\ how\ that\ relates\ to\ your\ findings.\textgreater{}}

\subsubsection{Outliers}

The original study reports the frequency of outliers in the text. There
was one outlier in the original study. In our replication we found
\ldots{}

\subsubsection{Relative bias}

Figures and tables report absolute bias. Results pertaining to relative
bias are only summarized in a more qualitative manner in text form. ``As
the number of categories increases, ML estimates gradually become less
biased and by five categories relative bias is always less than
10\%.(p.362)'' ``When the underlying distribution is non-normal, all
cat-LS parameter estimates take on a slightly positive bias (around
4\%), except when there are just two categories.'' (p.364) ``{[}B{]}ias
is almost never greater that 5\% with either method.

\subsubsection{Relative bias for robust standard error estimates}

``ML standard errors are from 8\% to 30\% (average = 15\%) smaller than
empirical standard errors when the sample size is small, and cat-LS
standard errors are from 3\% to 37\% (average 13\%) smaller than
empirical standard errors when the sample size is small.'' ``Cat-LS
produces better robust standard errors for factor loadings, and ML
produces better robust standard errors for factor correlations. This
finding isconsistent across number od categories.

\FloatBarrier
\section{Discussion}

\subsection{Replicability}

Due to the high amount of details in the original publication and the
corresponding supplemental materials the replication was straight
forward. The largest amount of time was spent ensuring that the methods
used for data generation and analysis did indeed correspond to what was
used in the original study. This is, however, in no way the fault of the
authors but rather due to limited documentation of the R packages used
fro replication. On the contrary the detailed description of the
implementation allowed for a close correspondence of methodology which
would have otherwise been left to guesswork.

A feature that deserves special praise with regards to facilitating
replicability is the high amount of documentation that the authors
dedicated to the generation of the simulated data as well as the
descriptives of the same. The ability to closely monitor the data
generation process and compare features of the simulated data to the
original study instilled a great deal of confidence in the replicators
and ensured that any potential deviations of results could not be
attributed to faulty interpretation and implementation of the data
generating mechanism.

Another feature that increased reproducibility was the structure of the
manuscript. The very first element of the method section was an overview
of the simulation factors. Readability was increased by listing each
factor as a separate bullet point. Subsequent sections detailed the
implementation of each simulation factor. A separate subheading for each
simulation factor made it easy to locate relevant information.

The detailed description of error handling procedures as well as error
descriptives \ldots{}

The large number of result tables presented in the supplemental material
is another exemplary reporting practice worth highlighting. While the
comparison of hundreds of table cells is not an easy endeavor and the
general interest in these tables likely limited it protects the authors
against any allegations of selective reporting and makes the assessment
of replicability possible.

A similar structure could be found for the performance measures which
were discussed in separate subsections separated by corresponding
heading. While very readable as is, we would have however preferred the
performance measures to be elaborated on as part of the method section
instead of the result section.

The introduction section included the presentation and discussion of
several closely related methods as well as findings from previous
studies investigating the same. Due to the large amount of information
surrounding highly similar methods and their implementation it took us
several readings of the introduction to feel confident about having
identified the version actually implemented in the study at hand. A
clearer separation of the implemented methods (e.g.~in a box) would have
facilitated isolating the relevant implementation details.

Finally, a major factor facilitating the reproduction process was the
availability of specialized SEM software in the R programming
environment. As R is frequently used for simulation studies
investigating SEM methodology we were able to build upon a code base
that was designed for this very purpose. While such specialized software
has the potential of huge time savings on the coding end and
additionally is likely to minimize coding errors on the part of the
replicator it consumes a significant amount of time to familiarize
oneself with the exact parameters underlying the tools. The
inexperienced user is at the mercy of the package documentation and the
occasional peek under the hood of a given function. Having a code base
from related simulation studies available would increase confidence in
using such tools and avoid some trial and error while familiarizing
oneself with the functionalities.

\texttt{\textless{}Provide\ a\ general\ statement\ of\ how\ you\ experienced\ the\ replication\ process.\ Was\ it\ easy?\ What\ made\ it\ easy\ or\ difficult?\textgreater{}}

\subsection{Replicator degrees of freedom}

\texttt{\textless{}Here\ you\ can\ discuss\ the\ replicator\ degrees\ of\ freedom.\ What\ could\ the\ authors\ have\ done\ to\ make\ it\ more\ clear?\ Do\ you\ think\ the\ replicator\ degrees\ of\ freedom\ are\ so\ extensive\ that\ they\ could\ influence\ the\ results?\textgreater{}}

\subsection{Equivalence of results}

\texttt{\textless{}How\ would\ you\ judge\ the\ overall\ equivalence\ of\ results?\ Are\ the\ orders\ of\ magnitude\ comparable?\ Are\ trends\ in\ the\ same\ direction?\ Would\ you\ draw\ the\ same\ conclusions\ as\ the\ authors\ based\ on\ your\ replication?\ Were\ some\ results\ not\ comparable\ because\ of\ insufficient\ figure\ resolution\ or\ labeling?\ Did\ the\ authors\ ommit\ some\ results\ which\ consequently\ cannot\ be\ compared?\textgreater{}}

\section{Acknowledgments}

\texttt{\textless{}Acknowledge\ the\ help\ of\ anyone\ who\ assisted\ you\ in\ the\ process\textgreater{}}

\section{Contributions}

Authors made the following contributions according to the CRediT
framework \url{https://casrai.org/credit/}

Anna Lohmann:

\begin{itemize}
\tightlist
\item
  Data Curation\\
\item
  Formal Analysis (lead)\\
\item
  Investigation\\
\item
  Software\\
\item
  Visualization (lead)\\
\item
  Writing - Original Draft Preparation\\
\item
  Writing - Review \& Editing
\end{itemize}

Arjan Huizing:

\begin{itemize}
\tightlist
\item
  Formal Analysis (supporting)\\
\item
  Investigation\\
\item
  Software (supporting)\\
\item
  Visualization (supporting)\\
\item
  Validation\\
\item
  Writing - Review \& Editing
\end{itemize}

\newpage

\section*{References}
\begingroup
\hphantom{x}
\setlength{\parindent}{-0.5in}
\setlength{\leftskip}{0.5in}

\hypertarget{refs}{}
\begin{CSLReferences}{1}{0}
\leavevmode\vadjust pre{\hypertarget{ref-burton_design_2006}{}}%
Burton, Andrea, Douglas G. Altman, Patrick Royston, and Roger L. Holder.
2006. {``The Design of Simulation Studies in Medical Statistics.''}
\emph{Statistics in Medicine} 25 (24): 4279--92.
\url{https://doi.org/10.1002/sim.2673}.

\leavevmode\vadjust pre{\hypertarget{ref-rougier_sustainable_2017-1}{}}%
Rougier, Nicolas P., Konrad Hinsen, Frédéric Alexandre, Thomas Arildsen,
Lorena A. Barba, Fabien C. Y. Benureau, C. Titus Brown, et al. 2017.
{``Sustainable Computational Science: The {ReScience} Initiative.''}
\emph{PeerJ Computer Science} 3 (December): e142.
\url{https://doi.org/10.7717/peerj-cs.142}.

\end{CSLReferences}

\FloatBarrier
\endgroup
\newpage

\section*{Appendix}

\subsection*{Additional result}

\texttt{\textless{}insert\ additional\ results\ not\ reported\ in\ the\ original\ article\ or\ results\ presented\ in\ an\ alternative\ way\textgreater{}}

\subsection{Code organization}

The code and the files associated are organized in the form of a
research compendium which can be found in the following git repository
\texttt{https://github.com/replisims/rhemtulla-2012}

\begin{verbatim}
## .
## +-- defs.tex
## +-- flowchart.PNG
## +-- Lato-Black.ttf
## +-- Lato-BlackItalic.ttf
## +-- Lato-Bold.ttf
## +-- Lato-BoldItalic.ttf
## +-- Lato-Italic.ttf
## +-- Lato-Regular.ttf
## +-- references.bib
## +-- Replication Report Rhemthulla et al 2012.Rmd
## +-- Replication-Report-Rhemthulla-et-al-2012.log
## +-- Replication-Report-Rhemthulla-et-al-2012.pdf
## +-- Replication-Report-Rhemthulla-et-al-2012.Rmd
## +-- Replication-Report-Rhemthulla-et-al-2012.tex
## +-- UbuntuMono-Bold.ttf
## +-- UbuntuMono-BoldItalic.ttf
## +-- UbuntuMono-Italic.ttf
## \-- UbuntuMono-Regular.ttf
\end{verbatim}

\begin{itemize}
\tightlist
\item
  \texttt{foldername}: contains
  \texttt{\textless{}insert\ description\textgreater{}}
\item
  \texttt{filename}: contains
  \texttt{\textless{}insert\ description\textgreater{}}
\item
  \ldots{}
\end{itemize}

\subsubsection*{Reproducibility Information}

This report was last updated on 2022-03-05 15:03:17. The simulation
replication was conducted using the following computational environment
and dependencies:

\FloatBarrier

\begin{verbatim}
## - Session info ---------------------------------------------------------------
##  setting  value
##  version  R version 4.1.2 (2021-11-01)
##  os       Windows 10 x64 (build 19043)
##  system   x86_64, mingw32
##  ui       RTerm
##  language (EN)
##  collate  English_United States.1252
##  ctype    English_United States.1252
##  tz       Europe/Berlin
##  date     2022-03-05
##  pandoc   2.14.0.3 @ C:/Program Files/RStudio/bin/pandoc/ (via rmarkdown)
## 
## - Packages -------------------------------------------------------------------
##  package        * version    date (UTC) lib source
##  assertthat       0.2.1      2019-03-21 [1] CRAN (R 4.1.2)
##  cachem           1.0.6      2021-08-19 [1] CRAN (R 4.1.2)
##  callr            3.7.0      2021-04-20 [1] CRAN (R 4.1.2)
##  cli              3.1.0      2021-10-27 [1] CRAN (R 4.1.2)
##  crayon           1.4.2      2021-10-29 [1] CRAN (R 4.1.2)
##  DBI              1.1.2      2021-12-20 [1] CRAN (R 4.1.2)
##  desc             1.4.0      2021-09-28 [1] CRAN (R 4.1.2)
##  devtools         2.4.3      2021-11-30 [1] CRAN (R 4.1.2)
##  digest           0.6.29     2021-12-01 [1] CRAN (R 4.1.2)
##  dplyr          * 1.0.8      2022-02-08 [1] CRAN (R 4.1.2)
##  ellipsis         0.3.2      2021-04-29 [1] CRAN (R 4.1.2)
##  evaluate         0.14       2019-05-28 [1] CRAN (R 4.1.2)
##  fansi            1.0.2      2022-01-14 [1] CRAN (R 4.1.2)
##  fastmap          1.1.0      2021-01-25 [1] CRAN (R 4.1.2)
##  fs               1.5.2      2021-12-08 [1] CRAN (R 4.1.2)
##  generics         0.1.2      2022-01-31 [1] CRAN (R 4.1.2)
##  glue             1.6.2      2022-02-24 [1] CRAN (R 4.1.2)
##  htmltools        0.5.2      2021-08-25 [1] CRAN (R 4.1.2)
##  knitr          * 1.37       2021-12-16 [1] CRAN (R 4.1.2)
##  lifecycle        1.0.1      2021-09-24 [1] CRAN (R 4.1.2)
##  magrittr         2.0.2      2022-01-26 [1] CRAN (R 4.1.2)
##  memoise          2.0.1      2021-11-26 [1] CRAN (R 4.1.2)
##  pillar           1.7.0      2022-02-01 [1] CRAN (R 4.1.2)
##  pkgbuild         1.3.1      2021-12-20 [1] CRAN (R 4.1.2)
##  pkgconfig        2.0.3      2019-09-22 [1] CRAN (R 4.1.2)
##  pkgload          1.2.4      2021-11-30 [1] CRAN (R 4.1.2)
##  prettyunits      1.1.1      2020-01-24 [1] CRAN (R 4.1.2)
##  processx         3.5.2      2021-04-30 [1] CRAN (R 4.1.2)
##  ps               1.6.0      2021-02-28 [1] CRAN (R 4.1.2)
##  purrr            0.3.4      2020-04-17 [1] CRAN (R 4.1.2)
##  R6               2.5.1      2021-08-19 [1] CRAN (R 4.1.2)
##  remotes          2.4.2      2021-11-30 [1] CRAN (R 4.1.2)
##  RepliSimReport   0.0.0.9000 2022-02-03 [1] Github (replisims/RepliSimReport@5f14003)
##  rlang            1.0.1      2022-02-03 [1] CRAN (R 4.1.2)
##  rmarkdown        2.11       2021-09-14 [1] CRAN (R 4.1.2)
##  rprojroot        2.0.2      2020-11-15 [1] CRAN (R 4.1.2)
##  rstudioapi       0.13       2020-11-12 [1] CRAN (R 4.1.2)
##  sessioninfo      1.2.2      2021-12-06 [1] CRAN (R 4.1.2)
##  stringi          1.7.6      2021-11-29 [1] CRAN (R 4.1.2)
##  stringr          1.4.0      2019-02-10 [1] CRAN (R 4.1.2)
##  testthat         3.1.1      2021-12-03 [1] CRAN (R 4.1.2)
##  tibble           3.1.6      2021-11-07 [1] CRAN (R 4.1.2)
##  tidyselect       1.1.1      2021-04-30 [1] CRAN (R 4.1.2)
##  usethis          2.1.5      2021-12-09 [1] CRAN (R 4.1.2)
##  utf8             1.2.2      2021-07-24 [1] CRAN (R 4.1.2)
##  vctrs            0.3.8      2021-04-29 [1] CRAN (R 4.1.2)
##  withr            2.4.3      2021-11-30 [1] CRAN (R 4.1.2)
##  xfun             0.29       2021-12-14 [1] CRAN (R 4.1.2)
##  xtable         * 1.8-4      2019-04-21 [1] CRAN (R 4.1.2)
##  yaml             2.2.1      2020-02-01 [1] CRAN (R 4.1.1)
## 
##  [1] C:/Users/alohmann/Documents/R/win-library/4.1
##  [2] C:/Program Files/R/R-4.1.2/library
## 
## ------------------------------------------------------------------------------
\end{verbatim}

The current Git commit details are:

\begin{verbatim}
## Local:    test C:/Users/alohmann/Dropbox/anna/projects_new/replisims/replications/rhemtulla-2012
## Remote:   test @ origin (https://github.com/replisims/rhemtulla-2012.git)
## Head:     [7b6ab82] 2022-02-28: A lot of stuff
\end{verbatim}


\end{document}
